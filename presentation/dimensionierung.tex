\section{Dimensionierungen}

\subsection{Widerstände}

\begin{frame}
	\frametitle{Ausgangsseite}
		\begin{block}{Richtwert für $R_C$}
			Der Richtwert gibt an, dass
			\[R_C < 0.3 \cdot R_L \]
			gelten muss.
		\end{block}
			$\Rightarrow R_C = 0.3 \cdot 22k \Omega = 6.6k \Omega$
			$\xrightarrow{E12} 5.6k \Omega$

		\[ R_E = \frac{\beta \cdot R_C - V_U \cdot R_{BE}}
		{V_U \cdot (1+\beta)} \approx \frac{R_C}{V_U} =
		\frac{5.6k \Omega}{5.62} = 
		996.4 \Omega \xrightarrow{E12} 1k \Omega \]
\end{frame}

\begin{frame}
	\frametitle{Ausgangsseite}
		\begin{block}{Regel für $\frac{R_C}{R_E}$}
			Die Regel besagt, dass 
			\[ \frac{R_C}{R_E} < 10 \]
			sein muss. Ist dies nicht gegeben, so muss zum
			Emitterwiderstand ein RC-Glied parallel geschaltet
			werden.	
		\end{block}
				
\end{frame}
