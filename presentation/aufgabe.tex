\section{Aufgabenstellung}

\subsection{Schaltung}

\begin{frame}
	\frametitle{Schaltung}
	\begin{figure}
		\centering
		\begin{circuitikz}[scale=0.75]\draw
			(-1,0) 	node[rground]{}
			(-1,3.5)to[sV=$U_e$]		(-1,0)
			(-1,3.5) to[R=$R_q$] 		(2,3.5)
			(2,3.5) to[C=$Cq$, -*] 		(4,3.5)
			(4,3.5) to[R=$R_2$] 		(4,0)
			(4,0) 	node[rground]{}
			(4,3.5) to[R=$R_1$] 		(4,7)
		%	(6,7) 	to[R=$R_C$] 		(npn.collector)
			(6,3.5) node[npn](npn){}
			(6,7)	to[R=$R_C$] (npn.collector)
			(4,3.5) to[short] 		(5,3.5) 
			(6,3) 	to[R=$R_E$] 		(6,0)
			(6,0) 	node[rground]{}
			(4,7) 	to[short, -*] 		(6,7)
			(6,7) 	to[short, -o] 		(10,7)
			(10,7)	node[anchor=west]{$U_B=+20V$}
			(6,4.5) to[short, *-] 		(6.5,4.5)
			(6.5,4.5) to[C=$C_A$] 		(8,4.5)
			(8,4.5) to[R=$R_L$] 		(8,0)
			(8,4.5) to[short, *-o] 		(10,4.5)
			(10,4.5) node[anchor=west]{$U_A$}
			(8,0) node[rground]{}
			% to[short, o-*](2,0)
			% to[L, l_=10mH, *-*](2,2)
			% to[R, l_=100k$\Omega$, *-o](0,2)
			% (2,2) to[short, *-*](4,2)
			% to[C=10nF, *-*](4,0)
			% (2,0) to[short, *-*](4,0)
			% (4,0) to[short, *-o](6,0)
			% (4,2) to[short, *-o](6,2)
			% (6,2) node[anchor=south]{C}
			% (6,0) node[anchor=north]{D}
			% (0,2) node[anchor=south]{A}
			% (-2,2) to[sI, l^=$f\neq kosnt.$](-2,0)
			% (-2,2) to[short](-1,2)
			% (-2,0) to[short](-1,0)
			% (6,0.2) to[open, v=$v(f)$] (6,1.8)
			;
		\end{circuitikz}
		\caption{AC-Verstärkerschaltung}
	\end{figure}
\end{frame}
